\documentclass{mytex} \usepackage{xcolor} \usepackage{fontawesome5} \usepackage{babel} \usepackage{tcolorbox} \usepackage{tikz} \usepackage{amsmath} % Pour les flèches \usepackage{setspace} \usepackage{hyperref} \title{usepackage{pgf-umlcd} \usepackage{array} \usepackage{booktabs} \usepackage{amsmath} % Pour les flèches \usepackage{setspace} \usepackage{hyperref} \usepackage{usepackage{laTEX_CMD_1__ \begin{document} __LATEX_CMD_2_ \title{LATEX_CMD_3__ \UE{SAÉ S2.04} __LATEX_CADEX_CADEX_CAD_CAD_CAD_CAD_CAD_CAD_CAD_CAD_CAD_CAD_CAD_CAD_CAD_CAD_CAD_CAD_CAD_CAD_CAD_CAD_CAD_CAD_CAD_EC_CAD_CAD_CAD_CAD_EC_EC_C_C_C It is not a statistical answer but a reminder of the context chosen. AAA \subsection {Introduction} This \textbf{report} will present the formatting of a database based on the Olympic Games (approximately 270,000 lines, 15 columns) in \emph{CSV} format. We will first import the data, then the \emph{ventiler}, or store them in more suitable tables in order to decompose the data at best. During the \emph{SQL} part, we will work on a \emph{PostgreSQL} server. \subsection{Gross data} We get the following file, weighing close to 40 MB. \inserfigure{img/file_explorer.png}{0.65cm}{Overview in the Explorer}{Label in the Figure} \inserfigure{img/file_apercu.png}{4cm}{Preview in the File}{Label in the Figure} \subsection{Observations} We then find that a lot of data is repeated and that's normal, because we are in a raw file. We can distinguish several possible tables: \begin{itemize} \item Athletes, \item Games, \item Sports, \item Delegations... Now let's move on to the design of this model! \subsection{The Data Logic Model (MLD)} \textbf{\emph{A Data Logic Model}} is a way to represent a table schema in both graphical and textual form. \tsec{Tables} We deduce, after reflection, close to 7 different tables: \inserfigure{img/version1/uml.png}{13cm}{Diagram UML}{fig::v1::uml} The logic used is: \begin{itemize} \item A primary key is named by the first letter of the name of the table, followed by \emph{"no"}, \item A primary key is underlined, \item A foreign key is preceded by a \emph{"__LATEX_CMD_11_"}. \end{itemize} \tsec{Raccourcis} One can think of Participation(\#ano, \#noc), which would be possible, but in reality this table is useless because an athlete who is part of a nation returns to the same as an athlete who participates for a nation (in Participation). In addition, an athlete can compete for one nation once, then for another another. \tsec{Precisions} Some explanations about namings that may not be explicit: \begin{itemize} \item A \textbf{\emph{NOC}} means \textbf{National Olympic Committee}. Example: United States --> USA. It is unique in the case of the Olympic Games. \item A discipline is a sub-genre of a sport. Example: Swimming --> 4x100m relay. It will be useful for the trials. \item For the rest, the reasons will be understood during the execution of the requests. \end{itemize} \section{Subsection} \subsection{Statement} This topic will deal with the Olympic Games, with the aim of analysing their data with a statistical approach. \subsection{Question Processing} We will therefore have to answer different statistical questions, using both data-based and statistical tools (formulas, graphs, etc.). Finally, in order to go further, we will imagine other statistical studies in \textbf{going further than the original subject}. All \textbf{ responses will be illustrated with appropriate graphs}. \subsection{Form of requests} By convention, requests and tables are written in English. The \\copy command, or \\o in SQL, will allow us to generate CSV files from the output of queries. Unfortunately, PostgreSQL forces to write everything on a single line without being able to identify, while it would be more readable to aerate requests, especially the densest ones. \begin{codeboxlang}{SQL} \copy (SELECT AVG(Participations.age) AS age_mean_global FROM Participations;) TO 'warage_global_age.csv' CSV HEADER \end{codeboxlang} \emph{Here, we request the global age and copy it to a file.} In this way, in the report, requests will be ventilated, without the command, and the files will have the above form. These files will be useful for better analysis of data, notably using the https://www.rawgraphs.io/ tool which, taking a dataset as input, allows its visualization in a multitude of formats. All these files are available in the root of this document. \section{Questions of the subject} This section will deal with questions relating to the subject. \subsection{Part 1} \tsec{Determine the 20 athletes with the most participations in O.J. (any sex combined)} \begin{codeboxlang}[title=20 athletes with the most participations]{SQL} SELECT Athletes.name, COUNT(Participations.ano) AS number_participations FROM Athletes JOIN Participation USING(ano) GROUP BY Athletes.name ORDER BY number_participations DESC LIMIT 20; \end{codeboxlang} \res{\insertfigure{img/version1/20ath.png}{12cm}{20 most active athletes}{fig:v1::20ath}} \subsection{Part 2} This part \textbf{statistic} will deal with a precise edition, and unique in terms of promotion, of the Olympic Games. Our edition will be the 1984 Summer Games in Los Angeles, USA. As the Summer Olympics are more monitored and with more testing, there will be more data to visualize than in winter, in order to have a more accurate and permissive statistical study in terms of options! \tsecnonum{Precision} In order to select the 1984 games in the table, each query will use the following view (this part is based only on the \emph{Participations}) \begin{codeboxlang}[title=Exclusive view at 1984]{SQL} CREATE VIEW Participations1984 AS ( SELECT Participations.* FROM Participations JOIN Events USING(eno) JOIN Games USING(gno) WHERE Games.games = '1984 Summer'; \end{codeboxlang} \tsec{Fill the statistical table} \begin{codeboxlang}[title=Complete statistical table]{SQL} SELECT Delegations.region, AVG(Participations.age) AS age_age, COUNT(DISTINCT Participations.ano) AS number_of_athletes, MIN(Participations.age) AS min_age, MAX(Participations) {Participations_engage} AS max_age USDelegations, MIN(Participations) AS number_of_athlet} It was their first participation in the Olympics, and they didn't have a medallist. \href{https://en.wikipedia.org/wiki/United\_Arab\_Emirates\_at\_the\_1984\_Summer__CMD_20__Olympics}{UAE in 1984} Note: Ages not reported are excluded when calculating the average but not the number of athletes. The average age is therefore not entirely representative, but it is more accurate than if the uninformed ages had been taken into account. \tsec{Compare the average age of the medallists and the average age of the participants (all genders)} First, we look for the age of the medallists and then the age of the participants. \emph{Participants therefore include everyone, including medallists.} \begin{codeboxlang}[title=Age average medallists]{SQL} SELECT AVG(Participations.age) AS age_mean_medails FROM Participations1984 WHERE medal IS NOT NULL; \end{codeboxlang} \res{24.24} \begin{codeboxlang}[title=Age average participant]{SQL} SELECT AVG(Participants.age) AS age_mean_participants FROM Participations1984; \end{codeboxlang} \res{comp{Stefqqqqqqqqqqqqqqqqqqqqqqqqqqqqqqqqqqqqqqqqqqqqqqqqqqqqqqqqqqqqqqqqqqqqqqqqqqqqqqqqqqqqqqqqqqqqqqqqqqqqqqqqqqqqqqqqqqqqqqqqqqqqqqqqqqqqqqqqqqqqqqqqqqqqqqqqqqqqqqqqqqqqqqqqqqqqqqqqqqqqqqqqqqqqqqqqqqqqqqqqqqqqqqqqqqqqqqqqqqqqqqqqqqqqqqqqqqqqqqqqqqqqqqqqqqqqqqqqqqqqqqqqqqqqqqqqqqqqqqqqqqqqqqqqqqqqqqqqqqqqqqqqqqqqqqqqqqqqqqqqqqqqqqqqqqqqqqqqqqqqqqqqqqqqqqqqqqqqqqqqqqqq The result will not be included each time to avoid overloading, as the statistical study will be present and will include these results. \begin{codeboxlang}[title=Exclusive view in 1992-2016 and the 5 countries]{sql} CREATE VIEW ParticipationsPart3 AS ( SELECT Participations.*, year, region FROM Participations JOIN Delegations USING(noc) JOIN Events USING(eno) JOIN Games USING(gno) WHERE Games.year BETWEEN 1992 AND 2016 AND region IN ('Australia', 'China', 'Italy', 'Spain', 'UK') ); \end{codeboxlang} \tsecnonum{Representation of evolutions} The results of the requests are saved in corresponding CSV files. These files are used as input to a small Python program (matplotlib) drawing a cloud of points corresponding to the data grouped by country and by game. This program also calculates a regression line in order to represent the overall evolution of the data according to their correlation. In some cases and depending on the parameters, we will have extrapolation or interpolation and will be judged in comment. This line does not apply to absolute truth and sometimes it will not be reliable. \textbf{It should be noted that the width of the boxes is different because between a Winter Game and a Summer Game, 1 and a half years ago, while in the forced case, 2 and a half years ago. } \tsec{Of the number of participants} \begin{codeboxlang}[title=Number of participants]{sql} SELECT region, games, COUNT(DISTINCT ano) AS contestants_number FROM ParticipationsPart3 GROUP BY region, games ORDER BY games ASC, contestants_number DESC; \end{codeboxlang} \inserfigure{img/version3/b1.png}{8.8cm}{Evolution pattern - B1}{fig::v3::b1} \obs{An interesting observation is that when the host country, like Australia in 2000 in Sydney or China in 2008 in Beijing, it concentrates much more participants (and it's quite logical because on the margins of these events a lot of sports investments are launched). Apart from these events, there is little difference in evolution.} \obs{Regression is probably strongly influenced by the fact that host countries concentrate more participants on this occasion. To designate a global evolution for these countries is therefore not entirely true to see these differences.} \tsec{Of the number of medallists} \begin{codeboxlang}[title=Number of medallists]{sql} SELECT region, games, COALESCE(COUNT(NULLIF(medal IS NULL, true)), 0) AS medals_number FROM ParticipationsPart3 GROUP BY region, games ORDER BY games ASC, medals_number DESC; \end{codeboxlang} \inserfigure{img/version3/b2.png}{8.8cm}{Evolution graph - B2}{fig:::v3:::b2} \obs{To say that there are more medallists is actually a statement depending on the countries being dealt with and the evolution of their sporting levels. Italy stagnated relatively while China increased significantly (in Beijing in 2008 in particular) as well as Australia (in Sydney in 2000).} \tsec{Of the number of women participating} \begin{codeboxlang}[title=Number of women participating by country and year]{sql} SELECT region, games, COUNT(DISTINCT ano) AS women_contestants_number FROM ParticipationsPart3 JOIN Athletes USING(ano) WHERE sex = 'F' GROUP BY region, games ORDER BY region ASC, games ASC, women_contestants_number DESC; \end{codeboxlang} \res{\inserfigure{img/version3/nbfemmes.png}{10cm}{Number of women participating}{fig::::v3::nbfemmes}} \inserfigure{img/version3/b3.png}{8.8cm}{Graphih of evolution - B3}{Fig more:yb more}{yb more}{b more:{b more} {b more} However, the winter games have fewer participants, so fewer participants. It is necessary to carefully analyze the points for this statement.} \tsec{of the proportion of women participating (in comparison with the proportion of men)} \begin{codeboxlang}[title=Proportion of participating women relative to men by country and year]{sql} SELECT region, games, COUNT(CASE WHEN sex = 'F' THEN 1 END) * 1.0 / NULLIF(COUNT(CASE WHEN sex = 'M' THEN 1 END), 0) AS women_over_men FROM ParticipationsPart3 JOIN Athlets USING(ano) GROUP BYregion, games ORDER BY region ASC, games ASC, women_over_men DESC; \end{codeboxlang} \inserfigure{img/version3/b4.png}{8.8cm}{Graphic of evolution - B4}{fig::v3:::b4} \obs{The average points are quite close; \end{codeboxlang} \inserfigure{img/version3/b4.png}{9}{Graphic of evolution - B4} \obs{obs{obs{obs{obs{obsqc_men are fairly close to men. The situation is, in 2016, in countries and according to evolution, egalitarian.} \obs{The next questions will deal with the number of people awarded medals and not medals: I thought I made a mistake so I looked a lot because the accounts did not follow but in reality, they are often collective events (rugby, relay, football...). Therefore, there are very often more medallists than medallists} \tsec{Of the proportion of medallists among women} \begin{codeboxlang}[title=Proportion of medallists among women]{sql} WITH WomenParticipation AS ( SELECT region, games, COUNT(DISTINCT ano) AS total_women FROM ParticipationsPart3 JOIN Athletes USING(ano) WHERE sex = 'F' GROUP BY region, games ), MedalsWomen AS ( SELECT region, games, CUNT(DISTINCT ano) AS medal_women FROM ParticipationsPart3 JOIN Athlees USING(ano) WHERE sex = 'F' AND medal IS NOT NULL GROUP BY region, games ) SELECT wp.region, wp.games, COALESCE(mw.medal_women}{Yyg_doy * 100.0 / NIGHTNIGHT,wwop. The regression nevertheless informs us that the proportion has increased by 0.05 over the given period.} \obs{However, the regression is not really reliable because there is too much difference between the values. A good regression usually has to concentrate a lot of points on its line and the points are too far away to allow it. In this situation, analyze the evolution makes little sense.} \tsec{From the proportion of women among the medallists} \begin{codeboxlang}[title=Proportion of women among the medallists]{sql} WITH AllCombinations AS ( SELECT DISTINCT region, games FROM ParticipationsPart3 ), MedalsPeople AS ( SELECT region, games, COUNT(DISTINCT ano) AS count_medal FROM ParticipationsPart3 WHERE medal IS NOT GROUP BY region, games ), MedalsWomen AS ( SELECT region, games, COUNT(DISTINCT ano) AS count_medal_women PART3 JOIN Athlets USING(ano) WHERE medal IS NOT GROUP BY region, games ), MedalsWomen AS (SELECT region, games, COUNT(DISTINCT ano) AS count_medal_women PART3 JOIN Athlets USING(ano) WHOME sex = 'F' AND medal IS NOT GROUP BY region, games} SELECT(DISTINCT ano) AS count_ms_womenenfgf, ogg_en_en_en_en_en_en_en_en_en_en_en_en_en_en_en_en_en_en_en_en_en_en_en_en_en_en_en_en_en_en_en_en_en_en_en_en_en_en_en_en_en_en_en_en_en_en_en_en_en_en_en_en_en_en_en_en_en_en_en_en_en_en_en_en_en_en_en_en_en_en_en_en_en_en_en_en_en_en_en_en_en_en_en_en_en_en_en_en_en_en_en_en_en_en_en_en_en_en_en_en_en_en_en_en_en_en_en_en_en_en_en_en_en_en_en_en_en_en_en_en_en_en_en_en_en_en_en_en_en_en_en The progression per year tends towards perfect equality (0.5), with for these countries and in this period, an end showing that the proportion of women among the medallists is equal to that of men.} \obs{The winter games are actually a bit poisoned gift of this data set. Indeed, there are 5 to 10 times fewer participants, so the results are very random (example: 8 participants, 0 medals, so 0 while even a medal would have made 0.125). For example, winter and summer proportions are counted at the same level, whereas in reality they involve quite different numbers of participants. The proportions to be studied should be the summer games because they have more data. However, if the subject of studies is evolution only, combining the two seasons is not a problem.} \obs{On these examples, we can generally see: \begin{itemize} \item An increase in the level of women \item A proportion becoming equal to that of men \item A better representation among the participants \end{itemize}} \section{To go further} \tsecnonum{Here, we will go further than the subject!} We will here base ourselves on all countries. \tsec{The impact of athlete weight change} I would have liked to make a request about athletes who have changed weight and the impact of this change on their weight, but the base admits only one single weight for each athlete. The same applies to size (which cannot logically change after 20 years). \tsec{The average age of sports medallists} It is often thought that you have to be young in sport. Here we will analyse which disciplines and sports require preparation from young people. * * * * * * * * * * * * * * * * * * * * * * * * * * * * * * * * * * * * * * * * * * * * * * * * * * * * * * * * * * * * * * * * * * * * * * * * * * * * * * * * * * * * * * * * * * * * * * * * * * * * * * * * * * * * * * * * * * * * * * * * * * * * * * * * * * * * * * * * * * * * * * * * * * * * * * * * * * * * * * * * * * * * * * * * * * * * * * * * * * * * * * * * * * * * * * * * * * * * * * * * * * * * * * * * * * * * * * * * * * * * * * * * * * * * * * * * * * * * * * * * * * * * * * * * * * * * * * * * * * * * * * * * * * * * * * * * * * * * * * * * * * * * * * * * * * * * * * * * * * * * * * * * * * * * * * * * * * * * * * * * * * * * * * * * * * * * * * * * * * * * * * * * * * * * * * * * * * * * * * * * * * * * * * * * * * * * * * * * * * * * * * * * * * * * * * * * * * * * * * * * * * * * * * * * * * * * * * * * * * * * * * * * * * * * * * * * * * * * * * * * * * * * * * * * * * * * * * * * * * * * * * * * * * * * * * * * * * * * * * * * * * * * * * * * * * * * * * * * * * * * * * * * * * * * * * \tsecnonum{The physics of medallists} \begin{codeboxlang}[title=Physics of medallists by sport]{sql} CREATE VIEW BodyMedalStats AS (SELECT sport, COALESCE(AVG(eight),0) AS avg_height, COALESCE(AVG(weight),0) AS avg_weight FROM ParticipationS JOIN Events USING(eno) JOIN Disciplines USING(dno) WHOE medal IS NOT NULL GROUP BY sport ; \end{codelang} \res{\inserfigure{img/version4/bodymedal.png}{10cm}{Physics of medallists by sport}{fig::v4::bodymedal} \tsecnonum} \begin{codeboxlang}[title=Physics of non-medal.png}{Savt} ISingight{Physics of medallists by sport}{sql} CREATE VIEWONUM} The opposite applies also to gymnastics. The other differences can also be sociological due to the way of life of the medallists, especially for weight, with tennis medallists of tables on average 1.73kg less heavy than the non-medallists. At the Games, however, someone who is not a medal-winning athlete is still a great athlete who has been able to compete in the biggest competitions of his sport, so it is already more likely that he meets the physical criteria of his discipline than anyone else (Basketball: the non-medalists make an average of 1m90).} \thank \end{document}