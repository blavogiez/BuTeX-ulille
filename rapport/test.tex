\documentclass{mytex} % On utilise la classe personnalisée

% Packages nécessaires pour les exemples
\usepackage{graphicx}
\usepackage{lipsum} % Pour du faux texte
\usepackage{hyperref}

\title{Exemple d'utilisation de mytex.cls}

\begin{document}

%------------ Paramètres du rapport ------------
\titre{Exemple de rapport avec mytex.cls}
\UE{SAÉ S2.04}
\sujet{Exploration des fonctionnalités}
\eleves{Jean \textsc{Dupont}, Clara \textsc{Martin}}
\enseignant{Dr. Pierre \textsc{Durand}}

%------------ Mises en page automatique ----------
\fairemarges
\fairepagedegarde
\tabledematieres

%------------ Début du contenu -------------------

\section{Introduction}
\lipsum[1]

\obs{Ce document utilise une classe LaTeX personnalisée pour améliorer la présentation des rapports.}

\comp{Comparé à la classe \texttt{article}, \texttt{mytex.cls} intègre directement les éléments de couverture, marges, et styles de boîtes.}

\res{Cette structure permet un gain de temps et une cohérence graphique dans les productions étudiantes.}

\section{Exemple de code SQL}

\begin{codeboxlang}[title=Création de la vue des médaillés]{sql}
CREATE VIEW BodyMedalStats AS (
SELECT sport, AVG(weight) AS avg_weight
FROM Participations
WHERE medal IS NOT NULL
GROUP BY sport
);
\end{codeboxlang}

\section{Exemple de code Python}

\begin{codeboxlang}[title=Traitement des données]{python}
def average(values):
	return sum(values) / len(values)
\end{codeboxlang}

\section{Section sans numérotation}
\tsecnonum{Zoom sur une fonctionnalité}
\lipsum[2]

%------------- Remerciement ----------------
\merci

\end{document}
