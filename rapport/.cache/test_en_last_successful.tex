\documentclass{mytex}
\policeprincipale{mathpazo}
\policesecondaire{mathpazo}



%-------------------- Informations sur le rapport ----------------------
% Remplir les informations qui seront utilisées sur la page de garde et dans les en-têtes/pieds de page
\titre{Développement}
\UE{Typographie}
\sujet{Projet de typographie avec \LaTeX}
\enseignant{Dr. LaTeX}
\eleves{Letudiant Joel \\ Letudiante Joelle}

%-------------------- Début du contenu du document ----------------------
\begin{document}


% Création de la page de garde
\fairepagedegarde

% Activation des en-têtes et pieds de page
\fairemarges

% Création de la table des matières

\tabledematieres


%======================================================================================
\section{Introduction and Basic Concepts}
%======================================================================================

The purpose of this document is to demonstrate the multiple features offered by the class \LaTeX{} personalized \texttt{ratio.cls}. Each section will explore a set of specific commands and environments to illustrate their use and visual rendering.

\info{
	This document is self-descriptive. \LaTeX{} which generates it is a direct example of the use of the class. Do not hesitate to consult the file \texttt{.tex} to see how each element is implemented.
}

We will start with basic elements such as lists and different levels of titles, before moving on to more complex topics. The correct use of French typographic quotes is done with the command \verb|\enquote{...}|Like this: \enquote{This is an example}.

\subsection{Smartlists and numbered}

The lists are stylized for better readability.

\tsecnonum{Example of chip list (itemize)}
\begin{itemize}
	\item First item in the list.
	\item Second element, which can extend over several lines if necessary to demonstrate indentation.
	\item Third item with a sublist:
	\begin{itemize}
		\item Sub-element A.
		\item Sub-element B.
	\end{itemize}
\end{itemize}

\tsecnonum{Example of numbered list (enumerate)}
\begin{enumerate}
	\item The first step is always the most important.
	\item The second follows logically.
	\item And so on, with a clear and bold numbering.
\end{enumerate}

\subsection{Structure of Titles}
The class defines a particular style for sections, sub-sections and sub-sub-sections, as you can see throughout this document. Here is an example of the hierarchy.

\subsubsection{This is a subtitle}
The style is more discreet to indicate a lower level of detail.


%======================================================================================
\section{Mathematics and Scientific Environments}
%======================================================================================
The class incorporates powerful tools for writing scientific content, including mathematics and theorems.

\subsection{Equations and Formulas}

The equations are numbered by section. Here are the equations of Maxwell, a classic example using the environment \texttt{Align}.
\begin{align}
	\nabla \cdot \mathbf{E} &= \frac{\rho}{\varepsilon_0} \label{eq:maxwell1} \\
	\nabla \cdot \mathbf{B} &= 0 \label{eq:maxwell2} \\
	\nabla \times \mathbf{E} &= -\frac{\partial \mathbf{B}}{\partial t} \label{eq:maxwell3} \\
	\nabla \times \mathbf{B} &= \mu_0 \left( \mathbf{J} + \varepsilon_0 \frac{\partial \mathbf{E}}{\partial t} \right) \label{eq:maxwell4}
\end{align}

\info{
	The Gauss equation (\ref{eq:maxwell1}) is fundamental in electromagneticsm.
}

\subsection{Theorems, Definitions and Remarks}
Predefined environments enable the scientific discourse to be structured.

\begin{definition}[Group]
	A group is a non-empty set $G$ with a law of internal composition $\ast$ which is associative, admits a neutral element and for which each element admits a symmetrical one.
\end{definition}

\begin{theorem}[Theorem of Lagrange]
	Si $H$ is a subgroup of a finished group $G$, then the order $H$ divides the order $G$.
\end{theorem}

\begin{exemple}
	All the relative integers $\mathbb{Z}$ with the addition is a group.
\end{exemple}

\begin{remarque}
	All these boxes share a consistent style for a pleasant reading.
\end{remarque}

\subsection{Scientific units}
The package \texttt{siunitx} is configured for French.

\tsec{Use of \texttt{siunitx}}
Planck constant \nomenclature{$h$}{Planck Constant} is about \num{6.626e-34}. The speed of light\nomenclature{$c$}{Speed of light in the vacuum} in the void is $c = \SI{299792458}{\meter\per\second}$.

%======================================================================================
\section{Visual Elements: Figures and Tables}
%======================================================================================

\subsection{Insertion of Figures}
Custom order \verb|\insererfigure|- Simplifies the addition of framed images.

% Utilisation de la commande personnalisée
\insererfigure{logos/logo.png}{4cm}{This is the main logo, inserted with our custom order \texttt{\textbackslash insertfigure}.}{_Principal Logo}

For more complex needs, such as sub-figures, standard environments always work.

\begin{figure}[H]
	\centering
	\begin{subfigure}{0.45\textwidth}
		\centering
		\includegraphics[width=0.8\linewidth]{logos/logo.png}
		\caption{First subfigure.}
		\label{fig:sub1}
	\end{subfigure}
	\hfill % Espace entre les deux figures
	\begin{subfigure}{0.45\textwidth}
		\centering
		\includegraphics[width=0.8\linewidth]{logos/logo_ECL.jpg}
		\caption{Second subfigure.}
		\label{fig:sub2}
	\end{subfigure}
	\caption{Example figure with two sub-figures using the package \texttt{subcaption}.}
	\label{fig:sousfigures}
\end{figure}


\subsection{Creation of Tables}
The tables are stylized with \texttt{booktabs} for a professional rendering.

\begin{table}[H]
	\centering
	\caption{Comparison of the characteristics of different languages.}
	\label{tab:langages}
	\begin{tabular}{l >{\raggedright\arraybackslash}p{4cm} c c}
		\toprule
		\textbf{Language} & \textbf{Main feature} & \textbf{Typing} & \textbf{Year} \\
		\midrule
		Python & Simplicity and readability & Dynamics & 1991 \\
		Java & \enquote{Write ounce, run anywhere} & Static & 1995 \\
		C++ & System performance and control & Static & 1985 \\
		\rowcolor{lightgray!50} % Exemple de couleur de ligne
		\multirow{-4}{*}{\rotatebox{90}{\textbf{Popular}}} & & & \\
		\bottomrule
	\end{tabular}
\end{table}


%======================================================================================
\section{Information Boxes and Code Listings}
%======================================================================================

\subsection{Information Boxes}
Several types of coloured boxes are available to highlight some information.

\res{
	It's the end of the experiment. \textbf{result} is positive and confirms our initial hypothesis.
}

\comp{
	Par \textbf{comparison}, approach A is 50\% faster than approach B, but consumes more memory.
}

\obs{
	Une \textbf{observation} important: the system becomes unstable when the temperature exceeds \SI{100}{\celsius}.
}

\warning{
	\textbf{Attention} : Never modify the kernel files directly, at the risk of corrupting the system.
}

\subsection{Code listings}
Environment \texttt{codeboxlang} allows to display code with a syntactic coloring adapted to the language.

\tsec{Example of Python Code}
\begin{codeboxlang}{python}
	# Simple script to greet the world def say_hello(name): """" This function displays a greeting message. """" print(e"Hello, {name}!) if __name__ == "_main__": say_hello("World")
\end{codeboxlang}

\tsec{Example Java code}
\begin{codeboxlang}{java}
	// File: HelloWorld.java public class HelloWorld {
		/** * The entry point of the program. */ public static void hand(String[] (args) {
			System.out.println("Hello, World from Java!"); 
		}
	}
\end{codeboxlang}

\tsec{Example SQL query}
\begin{codeboxlang}{sql}
	-- Select active users SELECT user_id, user_name, registration_date FROM users WHERE is_active = 1 ORDER BY registration_date DESC;
\end{codeboxlang}


%======================================================================================
\section{Nomenclature and References}
%======================================================================================

\subsection{Nomenclature}
Terms defined in the text with \verb|\nomenclature|They're gathered here.
\printnomenclature

\subsection{References and Hyperlinks}
The package \texttt{hyperref} is configured for internal and external links.
\begin{itemize}
	\item An internal link to the section on mathematics: see section \ref{sec:mathématiques-et-environnements-scientifiques}.
	\item An internal link to the logo figure: see figure \ref{fig:logo_principal}.
	\item An external link to the LaTeX project website: \url{https://www.latex-project.org/}.
\end{itemize}


%======================================================================================
\section{Conclusion}
%======================================================================================

This document has successfully explored much of the functionality of the class \texttt{ratio.cls}From layout to mathematical typography, graphic elements and code extracts, this class provides a robust and aesthetic framework for the writing of professional and academic reports.

\paragraph*{Perspectives}
It is obvious that the class \texttt{ratio.cls} is a powerful tool for the writing of professional and academic reports, offering a full range of features to ensure an aesthetic and structured layout. Moreover, its integration of mathematical elements allows to obtain well formatted documents in this specific field. Finally, the presence of code extracts, such as LaTeX listings or Python, offers an additional advance in terms of clarity and ease of understanding for a wide range of readers.

\merci

\end{document}






























